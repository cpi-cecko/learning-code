\documentclass[a4paper]{article}
\usepackage[utf8]{inputenc}
\inputencoding{utf8}
\usepackage[bulgarian]{babel}
\selectlanguage{bulgarian}

\usepackage{amsmath}
\usepackage[margin=1in]{geometry}
\title{Домашно 1 по \\* ``Ламбда смятане и теория на доказателствата''}
\author{Цветан Цветанов, студент 3 курс, ИС, ФМИ}

\begin{document}

\maketitle
\thispagestyle{empty}
\newpage

\section*{Задача 1}

Докажете, че двете дефиниции за числа на Хеминг са еквивалентни:

\begin{align*}
H_{1} = \{ x \mid p/x, p \in \{2, 3, 5\}\} \iff \quad H_{2}: \quad & 1)\; 1 \in H_{2} \\
                                                                   & 2)\; h \in H_{2} \implies 2h, 3h, 5h \in H_{2}
\end{align*}

Д-во:
\begin{align*}
& 1)\; 1 \in H_{2} \implies 1 \in H_{1}, \text{ защото 1 дели 2, 3 и 5 } \\
& 2)\; h \in H_{2} \implies \\ 
      & \qquad 2h \in H_{2}, \text{ но } 2/2h = 1/h \implies h \in H_{1} \text{ или} \\
      & \qquad 3h \in H_{2}, \text{ но } 3/3h = 1/h \implies h \in H_{1} \text{ или} \\
      & \qquad 5h \in H_{2}, \text{ но } 5/5h = 1/h \implies h \in H_{1} \\
      & \implies H_{2} \subseteq H_{1} \\
& \text{Аналогично } H_{1} \subseteq H_{2} \implies H_{1} = H_{2} \\
Q.E.D.
\end{align*}

\section*{Задача 2}

Да се докаже, че с индукция по дефиниция се построява графиката на тотална функция \\
$f : X \rightarrow Y, F$

Д-во:
\begin{align*}
& dom F = \{ x | f(x) \in Y, x \in X \} \implies F \subseteq X \implies F \in G_f \\
Q.E.D.
\end{align*}

\section*{Задача 3}

Да се провери, че във всяка съждителна формула $F$, има еднакъв брой затварящи и отварящи скоби.

Дефинираме функцията $m : X \rightarrow \{0, 1\}$:

\[ m(x) =
\begin{cases}
  1    & \quad x = P \text{ или } x = (P)\\
  m(A) & \quad x = \neg A \text{ или } x = (\neg A)\\
  m(A)\;\&\&\;m(B) & \quad x = A \sigma B \text{ или } x = (A \sigma B)\\
  0 & \quad \text{ иначе }
\end{cases}
\]

F съждителна формула:
\begin{align*}
&  1) x \in V \implies x \in F\\
&  2) \phi, \psi \text{ са съждителни формули } \implies (\neg \phi), (\phi \land \psi), (\phi \lor \psi) \in F
\end{align*}

По дефиницията на $F$, за всяко $F$, $m \text{ връща } 1 \implies F \subseteq X$.

Q.E.D.

\section*{Задача 4}

Дефинирайте $M^x_y$ заместване на променливата $x$ с променливата $y$ в $M$, ако $x \in BV(M), y \notin BF(M) \cup FV(M)$, а $M$ е $\lambda$-терм.

\begin{align*}
& 1) M \equiv x \implies M^x_y := y \\
& 2) M \equiv (M_1, M_2) \implies M^x_y := ({M_1}^x_y)({M_2}^x_y) \\
& 3) M \equiv \lambda_xM' \implies M^x_y := \lambda_yM'^x_y
\end{align*}

\end{document}
